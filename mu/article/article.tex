\documentclass[prb,11pt,tightenlines,twocolumn,aps]{revtex4-1}

% preamble:

\usepackage{amsmath}    % need for subequations
\usepackage{graphicx}   % need for figures
\usepackage{verbatim}   % useful for program listings
\usepackage{color}      % use if color is used in text
\usepackage{subfigure}  % use for side-by-side figures
\usepackage{hyperref}   % use for hypertext links, including those to external
                        % documents and URLs 
\usepackage{blindtext}  % fill text
\usepackage{color}
% \usepackage{showframe}
\usepackage[inline]{showlabels}

\raggedbottom           % don't add extra vertical space

\begin{document}

\title{Pure Spin Current Injection in Hydrogenated Graphene Structures}
\author{Reinaldo Zapata-Pe\~na\textsuperscript{1},
        Bernardo S. Mendoza\textsuperscript{1},
        Anatoli I. Shkrebtii\textsuperscript{2}}
\affiliation{\textsuperscript{1}Centro de Investigaciones en \'Optica, Le\'on,
Guanajuato 37150, M\'exico}
\affiliation{\textsuperscript{2}University of Ontario, Institute of Technology,
Oshawa, ON, L1H 7L7, Canada}

\date{\today}

\begin{abstract}
\blindtext
\end{abstract}

\maketitle

%%%%%%%%%%%%%%%%%%%%%%%%%%%%%%%%%%%%%%%%%%%%%%%%%%%%%%%%%%%%%%%%%%%%%%%%%%%%%%
%%%%%%%%%%%%%%%%%%%%%%%%%%%%%%%  INTRODUCTION %%%%%%%%%%%%%%%%%%%%%%%%%%%%%%%%
%%%%%%%%%%%%%%%%%%%%%%%%%%%%%%%%%%%%%%%%%%%%%%%%%%%%%%%%%%%%%%%%%%%%%%%%%%%%%%

\section{Introduction}
\blindtext
\begin{figure}[ht!]
    \centering
    \subfigure[\ $xz$ plane view]
    {\includegraphics[width=\linewidth]{figures/altstruc2}}
    \label{fig:alt-struc-xz}
    \\
    \subfigure[\ $xy$ plane view]
    {\includegraphics[width=\linewidth]{figures/altstruc1}}
    \label{fig:alt-struc-xy}
    \caption{Alt structure.}
    \label{fig:alt-struc}
\end{figure}
\begin{figure}[ht!]
    \centering
    \subfigure[\ $xz$ plane view]
    {\includegraphics[width=\linewidth]{figures/upstruc2}}
    \label{fig:up-struc-xz}
    \\
    \subfigure[\ $xy$ plane view]
    {\includegraphics[width=\linewidth]{figures/upstruc1}}
    \label{fig:up-struc-xy}
    \caption{Up structure}
    \label{fig:up-struc}
\end{figure}
\blindtext
\blindtext

\blindtext

%%%%%%%%%%%%%%%%%%%%%%%%%%%%%%%%%%%%%%%%%%%%%%%%%%%%%%%%%%%%%%%%%%%%%%%%%%%%%%
%%%%%%%%%%%%%%%%%%%%%%%%%%%%%%%%%%% THEORY %%%%%%%%%%%%%%%%%%%%%%%%%%%%%%%%%%%
%%%%%%%%%%%%%%%%%%%%%%%%%%%%%%%%%%%%%%%%%%%%%%%%%%%%%%%%%%%%%%%%%%%%%%%%%%%%%%

\section{Theory} % (fold)
\label{sec:theory}
The equation for $\mathcal{V}^{\mathrm{ab}}$ for normal incidence in the $xy$
plane with a polarization angle $\alpha$ is given by

\begin{widetext}
\begin{align}
\mathcal{V}^{\mathrm{ab}} (\omega) 
&= \frac{2}{\hbar}
\frac{\mu^{\mathrm{abxx}}(\omega)
E^{2}(\omega)\cos^{2}(\alpha) + 
\mu^{\mathrm{abyy}}(\omega)
E^{2}(\omega)\sin^{2}(\alpha) + 
2\mu^{\mathrm{abxy}}(\omega)
E^{2}(\omega)\cos(\alpha)\sin(\alpha)}
{\xi^{\mathrm{xx}}(\omega)
E^{2}(\omega)\cos^{2}(\alpha) + 
\xi^{\mathrm{yy}}(\omega)
E^{2}(\omega)\sin^{2}(\alpha)},
\nonumber \\
&= \frac{2}{\hbar}
\frac{\mu^{\mathrm{abxx}}(\omega)\cos^{2}(\alpha) + 
\mu^{\mathrm{abyy}}(\omega)\sin^{2}(\alpha) + 
\mu^{\mathrm{abxy}}(\omega)\sin(2\alpha)}
{\xi^{\mathrm{xx}}(\omega)\cos^{2}(\alpha) + 
\xi^{\mathrm{yy}}(\omega)\sin^{2}(\alpha)}.
\label{eq:vab}
\end{align}
\end{widetext}

For an angle $\alpha = \frac{\pi}{4}$ this expression can be reduced to 
\begin{align}
\mathcal{V}^{\mathrm{ab}} (\omega)
&= \frac{2}{\hbar}
\frac{\mu^{\mathrm{abxx}}(\omega) + \mu^{\mathrm{abyy}}(\omega) + 
2\mu^{\mathrm{abxy}}(\omega)}
{\xi^{\mathrm{xx}}(\omega) + \xi^{\mathrm{yy}}(\omega)}.
\label{eq:vab-90deg}
\end{align}

%%%%%%%%%%%%%%%%%%%%%%%%%%%%%%%%%%%%%%%%%%%%%%%%%%%%%%%%%%%%%%%%%%%%%%%%%%%%%%
%%%%%%%%%%%%%%%%%%%%%%%%%%%% Theory: Fixing vel %%%%%%%%%%%%%%%%%%%%%%%%%%%%%%

\subsection{Fixing velocity.}\label{sec:theory-fixvel}
Considering that we have 2D structures we fixed the velocity in the $xy$ plane
along $x$ and $y$ directions and we define $|\mathcal{V}^{\mathrm{a}}|$ as
\begin{equation}\label{eq:va-mag}
|\mathcal{V}^{\mathrm{a}}| = 
\sqrt {
(\mathcal{V}^{\mathrm{ax}})^{2} +
(\mathcal{V}^{\mathrm{ay}})^{2} +
(\mathcal{V}^{\mathrm{az}})^{2} 
},
\end{equation}
and the corresponding polar and azimuthal angles $\theta$ and $\varphi$ as
\begin{align}
\theta  =& \cos^{-1} \left( \frac{\mathcal{V}^{\mathrm{az}}}
{|\mathcal{V}^{\mathrm{a}}|} \right),
& 0 \leq &\theta \leq \pi, 
\label{eq:polar-ang}
\\
\varphi =& \tan^{-1} \left( \frac{\mathcal{V}^{\mathrm{ay}}}
{\mathcal{V}^{\mathrm{ax}}} \right),
& 0 \leq &\varphi \leq 2\pi.
\label{eq:azimuthal-ang} 
\end{align}

%%%%%%%%%%%%%%%%%%%%%%%%%%%%%%%%%%%%%%%%%%%%%%%%%%%%%%%%%%%%%%%%%%%%%%%%%%%%%%
%%%%%%%%%%%%%%%%%%%%%%%%%%%% Theory: Fixing spin %%%%%%%%%%%%%%%%%%%%%%%%%%%%%

\subsection{Fixing spin}\label{sec:theory-fixspin}
In a similar way we can fix in the $xy$ plane the spin direction along the $x$,
$y$, and $z$ directions and then define the magnitude of the spin velocity $|\mathcal{V}_{\sigma^{\mathrm{b}}}|$ in a fixed angle $\gamma_{b}$
\begin{align}
|\mathcal{V}_{\sigma^{\mathrm{b}}}| 
&=
\sqrt{
(\mathcal{V}^{\mathrm{ax}})^{2} +
(\mathcal{V}^{\mathrm{ay}})^{2} 
}, \\
\gamma_{\mathrm{b}} 
&=
\tan^{-1} \left( \frac{\mathcal{V}^{\mathrm{ay}}}
{\mathcal{V}^{\mathrm{ax}}} \right),
\end{align}
where the angle is measured in the counter-clockwise direction from the positive
$x$ axis.

% section theory (end)

%%%%%%%%%%%%%%%%%%%%%%%%%%%%%%%%%%%%%%%%%%%%%%%%%%%%%%%%%%%%%%%%%%%%%%%%%%%%%%
%%%%%%%%%%%%%%%%%%%%%%%%%%%%%%%%%% RESULTS %%%%%%%%%%%%%%%%%%%%%%%%%%%%%%%%%%%
%%%%%%%%%%%%%%%%%%%%%%%%%%%%%%%%%%%%%%%%%%%%%%%%%%%%%%%%%%%%%%%%%%%%%%%%%%%%%%

\section{Results} % (fold)
\label{sec:results}


We preset the results for $\mathcal{V}^{\mathrm{ab}}$ for the
C$_{16}$H$_{8}$-alt and C$_{16}$H$_{8}$-up structures being both
noncentrosymmetric semi-infinite carbon systems with 50\% hydrogenation in
different arrangements. The \emph{alt} system has alternating hydrogen atoms on
the upper and bottom sides of the carbon sheet, while the \emph{up} system has H
only on the upper side. We take the hexagonal carbon lattice to be on the $xy$
plane for both structures, and the carbon-hydrogen bonds on the perpendicular
$xz$ plane, as depicted in Figs.
\ref{fig:alt-struc} and \ref{fig:up-struc}.

Using the ABINIT code \cite{gonzeCPC09} we calculated the self- consistent
ground state and the Kohn-Sham states using density functional theory in the
local density approximation (DFT-LDA) with a planewave basis. We used
Hartwigsen- Goedecker-Hutter (HGH) relativistic separable dual-space Gaussian
pseudopotentials \cite{hartwigsenPRB98} including the spin-orbit interaction
for calculating $\mathcal{V}^{\mathrm{a}}(\omega)$.

The convergence parameters for the calculations of our results corresponding to
the \emph{alt} and \emph{up} structures are cutoff energies of 65\,Ha and
40\,Ha, respectively. The energy eigenvalues and matrix elements were
calculated using 14452 $\mathbf{k}$ points and 8452 $\mathbf{k}$ points in the
irreducible Brillouin zone (IBZ) and present LDA energy band gaps of 0.72\,eV
and 0.088\,eV, respectively for the \emph{alt} and \emph{up} structures. As
mentioned in
\cite{zapataPSB2016}, using DFT the LDA is only one method of many other that
can be used to calculate the electronic structure of materials. Also it is
known that all methods predict a different band gap than the obtained in the
experiment. A correction for the band gap energy value can be calculated by
other \emph{ab-initio} methods such as the GW approximation \cite{onidaRMP02}
being this outside the scope of this paper.



\begin{table}[tb]
\center
\begin{tabular}{ccccc}\\
\hline
Layer & Atom & \multicolumn{3}{c}{Position [\AA]} \\
\cline{3-5}
No. & type & $x$ & $y$ & $z$  \\
\hline
1 & H &  -0.61516 &  -1.42140 & \ 1.47237 \\
2 & C &  -0.61516 &  -1.73300 & \ 0.39631 \\
3 & C & \ 0.61516 & \ 1.73300 & \ 0.15807 \\
4 & C & \ 0.61516 & \ 0.42201 &  -0.15814 \\
5 & C &  -0.61516 &  -0.37396 &  -0.39632 \\
6 & H &  -0.61516 &  -0.68566 &  -1.47237 \\
\hline
\end{tabular}
\caption{Unit cell of \emph{alt} structure. Layer division, atom types and
positions for the \emph{alt} structure. The structure unit cell was divided in
six layers corresponding each one to atoms in different $z$ positions. The
corresponding layer atom position is depicted in Fig. \ref{fig:alt-struc} with
the corresponding number of layer.}
\label{tab:alt-unitcell}
\end{table}

The structures presented here where divided into layers to analyze the he
layer-by-layer contribution for $\mathcal{V}^{\mathrm{ab}}$ response. The
\emph{alt} structure was divided in six layers corresponding the first one to
the top hydrogen atoms, from the second to the forth to carbon atoms in
different $z$ positions, and the sixth and last one to the bottom hydrogen
atoms. The \emph{up} structure was divided into two layers, the first one
comprised by the top hydrogen atoms and the second by the carbon atoms. The
layer divisions and atom positions for the unit cells are shown in Tables
\ref{tab:alt-unitcell} and \ref{tab:up-unitcell}.

\begin{table}[tb]
\center
\begin{tabular}{ccccc}\\
\hline
Layer & Atom & \multicolumn{3}{c}{Position [\AA]} \\
\cline{3-5}
No. & type & $x$ & $y$ & $z$  \\
\hline
1 & H & -0.61516 & -1.77416 &  0.73196 \\
1 & H &  0.61518 &  0.35514 &  0.73175 \\
2 & C & -0.61516 & -1.77264 & -0.49138 \\
2 & C & -0.61516 & -0.35600 & -0.72316 \\
2 & C &  0.61516 &  0.35763 & -0.49087 \\
\hline
\end{tabular}
\caption{Unit cell of \emph{up} structure. Layer division, atom types and
positions for the \emph{up} structure. The structure unit cell was divided in
two layers corresponding to hydrogen and carbon atoms.The corresponding layer
atom position is depicted in Fig. \ref{fig:up-struc} with the corresponding
number of layer.}
\label{tab:up-unitcell}
\end{table}



%%%%%%%%%%%%%%%%%%%%%%%%%%%%%%%%%%%%%%%%%%%%%%%%%%%%%%%%%%%%%%%%%%%%%%%%%%%%%%
%%%%%%%%%%%%%%%%%%%%%%%%%%% Res: fixin vel Alt  %%%%%%%%%%%%%%%%%%%%%%%%%%%%%%%

\subsection{Fixing velocity} % (fold)
\label{sec:res-fixvel}

\begin{figure}[tb]
    \centering
    \subfigure[\ $|\mathcal{V}^{\mathrm{x}}|$ for \emph{alt} structure 
    \label{fig:alt-3d-vx}]
    {\includegraphics[width=\linewidth]{altplots/alt-3d-vxb}}
    \\
    \subfigure[\ $|\mathcal{V}^{\mathrm{x}}|$ for \emph{alt} structure 
    \label{fig:alt-3d-vy}]
    {\includegraphics[width=\linewidth]{altplots/alt-3d-vyb}}
    \caption{$|\mathcal{V}^{\mathrm{x}}|$ response for C$_{16}$H$_{8}$-alt
    structure. The maximum response zone is localized for an energy range from
    0.90\,eV to 0.93\,eV. 145$^{\circ}$ and for a polarization angle of the
    incoming beam from 120$^{\circ}$ to 150$^{\circ}$.}
    \label{fig:alt-vab-mag}
\end{figure}

\begin{figure}[tb]
    \centering
    \subfigure[\ $|\mathcal{V}^{\mathrm{x}}|$ \label{fig:alt-vx-comp-rtp}]
    {\includegraphics[width=\linewidth]{altplots/alt-vxb-rtp-m}}
    \\
    \subfigure[\ $|\mathcal{V}^{\mathrm{y}}|$ \label{fig:alt-vy-comp-rtp}]
    {\includegraphics[width=\linewidth]{altplots/alt-vyb-rtp-m}}
    \caption{Most intense responses of $|\mathcal{V}^{\mathrm{x}}|$ and
    $|\mathcal{V}^{\mathrm{y}}|$ and the corresponding three components for the
    \emph{alt} structure. Both maxima where obtained for a polarization
    angle $\alpha=145^{\circ}$. }
    \label{fig:alt-vab-comp-rtp}
\end{figure}

For the \emph{alt} structure we analyzed the energy range of energy from
0.6\,eV to 1.0\,eV where we found the most intense response for
$\mathcal{V}^{\mathrm{ab}}$ and $|\mathcal{V}^{\mathrm{a}}|$. In Fig. 
\ref{fig:alt-vab-mag} we present the $|\mathcal{V}^{\mathrm{a}}|$ spectra
resulting from evaluate Eq. \eqref{eq:va-mag} using different polarization
angles $\alpha$ in Eq. \eqref{eq:vab} for the C$_{16}$H$_{8}$-alt structure. We
can see that the onset of the response is when the energy of the incoming light
is the same of the gap energy.
%
From this picture we can see that for the zone between the energy range of
0.90\,eV-0.93\,eV and polarization angles between 120$^{\circ}$ and
150$^{\circ}$ is the zone of the absolute maximum response for both,
$|\mathcal{V}^{\mathrm{x}}|$ and $|\mathcal{V}^{\mathrm{y}}|$. Also there is
another zone of interest for energies from 0.70\,eV to 0.74\,eV where a local
maximum is obtained.
From Fig. \ref{fig:alt-3d-vx} we have that $|\mathcal{V}^{\mathrm{x}}|$reaches
values of 30\,Km/s for the first zone mentioned before and 20\,Km/s for
the second one. 
% %%%%%%%%%%%%%%%%
% %% Starting description of ALT |V^{a}|, components and RTP
% %%%%%%%%%%%%%%%%
We also found that the absolute maximum of the response is obtained when the
polarization angle is $\alpha = 145^{\circ}$. 
%
In the top frames of Figs. \ref{fig:alt-vx-comp-rtp}  and 
\ref{fig:alt-vy-comp-rtp} we present the results for
$|\mathcal{V}^{\mathrm{x}}|$ and $|\mathcal{V}^{\mathrm{y}}|$ fixing the
polarization angle to $145^{\circ}$ for the \emph{alt} structure vs the photon
energy and the corresponding azimuthal $\theta$ and polar $\varphi$ angles.
%
Also in the bottom frames of Figs. \ref{fig:alt-vx-comp-rtp} and 
%
\ref{fig:alt-vy-comp-rtp} we present the decomposition of
$|\mathcal{V}^{\mathrm{x}}|$ and $|\mathcal{V}^{\mathrm{y}}|$ in the
corresponding $\mathcal{V}^{\mathrm{xx}}$, $\mathcal{V}^{\mathrm{xy}}$,
$\mathcal{V}^{\mathrm{xz}}$ and
$\mathcal{V}^{\mathrm{yx}},\mathcal{V}^{\mathrm{yy}}$
$\mathcal{V}^{\mathrm{yz}}$ components for the fixed polarization angle.
% %%%%%%%%%%%%%%%%
% %% Description of ALT |V^{x}|, components and RTP
% %% files: vab-rtp.sm_0.03_x_14452_65-spin_scissor_0_Nc_32_ang_145
% %%        alt-plots/alt-vab-rtp.gp
% %%%%%%%%%%%%%%%%
Making the analysis for the components and angles for
$|\mathcal{V}^{\mathrm{x}}|$ depicted in Fig. \ref{fig:alt-vx-comp-rtp} we can
see that for the energy range from 0.70\,eV to 0.74\,eV all the $xx$, $xy$, and
$xz$ components contribute with almost the same intensity giving a total spin-
velocity of 19.3\,Km/s and spin polar and azimuthal angles $\varphi =
45.8^{\circ}$ and $\theta=40.7^{\circ}$.
%
In the other hand, for the energy range from 0.88\,eV to 0.95\,eV there is a
major contribution coming from the $\mathcal{V}^{\mathrm{xz}}$ component
resulting in a spin-velocity magnitude of 30.9\,Km/s being this magnitude the
most intense for $|\mathcal{V}^{\mathrm{x}}|$. In this case the polar angle is
$\varphi=153.8^{\circ}$ and the spin angle over the $xy$ plane have is $\theta=290.4^{\circ}$.
%
Also we notice that for the range of 0.70-0.74\,eV all the contributions are
positive while for the range of 0.88-0.95\,eV the $xx$ component remains
positive but the components $xy$ and $xz$ change in direction. This is due to
\textcolor{red}{\large a change in the spin polarization}.
% %%%%%%%%%%%%%%%%
% %% Description of ALT |V^{y}|, components and RTP
% %% files: vab-rtp.sm_0.03_y_14452_65-spin_scissor_0_Nc_32_ang_145
% %%        alt-plots/alt-vab-rtp.gp
% %%%%%%%%%%%%%%%%
Making now the analysis for $|\mathcal{V}^{\mathrm{y}}|$ depicted in Fig.
\ref{fig:alt-vy-comp-rtp} we have that for the energy range from 0.70\,eV to
0.74\,eV the $yz$ component have a more intense response than $yx$ and $yy$
components and they give a total spin-velocity of 51.9\,Km/s and result in polar
angle $\varphi=140.7^{\circ}$ and spin azimuthal angle of
$\theta=221.5^{\circ}$.
%
For the energy range from 0.88\,eV to 0.95\,eV all three components have similar
intensities resulting in a spin-velocity magnitude of
$|\mathcal{V}^{\mathrm{y}}| =$ 52.3\,Km/s being this response 1.7 times mores
intense than the most intense response of $|\mathcal{V}^{\mathrm{x}}|$. The
corresponding polar and azimuthal angles for $|\mathcal{V}^{\mathrm{y}}|$ in
this energy range are $\theta = 129.0 ^{\circ}$ and $\varphi = 228.9 ^{\circ}$.
%
Now we have that the three components of $|\mathcal{V}^{\mathrm{y}}|$ are
negative for the energy range from 0.60\,eV to 1.0\,eV \textcolor{red}{keeping
the same spin polarization for all this range.} 
%
\begin{figure}[tb]
    \centering
    \subfigure[\ $|\mathcal{V}^{\mathrm{x}}|$ \label{fig:alt-vx-lay}]
    {\includegraphics[width=\linewidth]{figures/blank}}
    \\
    \subfigure[\ $|\mathcal{V}^{\mathrm{y}}|$ \label{fig:alt-vy-lay}]
    {\includegraphics[width=\linewidth]{figures/blank}}
    \caption{Layer-by-layer contribution of $|\mathcal{V}^{\mathrm{x}}|$ and
    $|\mathcal{V}^{\mathrm{y}}|$ for a polarization angle $\alpha=145^{\circ}$
    for the \emph{alt} structure.}
    \label{fig:alt-vab-lay}
\end{figure}
\begin{figure}[tb]
    \centering
    \includegraphics[width=\linewidth]{altplots/alt-vyz-layers}
    \caption{Layer-by-layer contribution of $\mathcal{V}^{\mathrm{yz}}$ for the
     \emph{alt} structure.}
    \label{fig:alt-vyz-lay}
\end{figure}

% %%%%%%%%%%%%%%%%
% %% Description of ALT layer |V^{ab}|
% %% files: 
% %%%%%%%%%%%%%%%%
In Fig. \ref{fig:alt-vab-lay} we show the layer-by-layer contribution of
$|\mathcal{V}^{\mathrm{ab}}|$  for the \emph{alt} structure. The corresponding
layer division and atom types and positions are presented in Table 
% 
\ref{tab:alt-unitcell} and depicted in Fig. \ref{fig:alt-struc}. In the layer-
by-layer contribution for $|\mathcal{V}^{\mathrm{x}}|$ (Fig. \ref{fig:alt-vx-lay})
we have that
% 
\textcolor{red}{\ldots}
% 
Also, for the layer-by-layer contribution for $|\mathcal{V}^{\mathrm{y}}|$ (Fig.
\ref{fig:alt-vy-lay}) we have that
% 
\textcolor{red}{\ldots}
%
% %%%%%%%%%%%%%%%%
% %% Description of ALT layer |V^{yz}|
% %% files:  v.sm_0.03_yz_14452_65-spin_scissor_0_Nc_32_ang_145
% %%         calv.sm_0.03_yz_14452_1_65-spin_scissor_0_Nc_32_ang_145
% %%         calv.sm_0.03_yz_14452_2_65-spin_scissor_0_Nc_32_ang_145
% %%         calv.sm_0.03_yz_14452_3_65-spin_scissor_0_Nc_32_ang_145
% %%         calv.sm_0.03_yz_14452_4_65-spin_scissor_0_Nc_32_ang_145
% %%         calv.sm_0.03_yz_14452_5_65-spin_scissor_0_Nc_32_ang_145
% %%         calv.sm_0.03_yz_14452_6_65-spin_scissor_0_Nc_32_ang_145
% %%         alt-plots/alt-vab-layers.gp
% %%%%%%%%%%%%%%%%
From the bottom panels of Fig. \ref{fig:alt-vab-comp-rtp} we can see that for
the \emph{alt} structure the most intense component of
$|\mathcal{V}^{\mathrm{x}}|$ and $|\mathcal{V}^{\mathrm{y}}|$ corresponds to
$\mathcal{V}^{\mathrm{yz}}$ which has a value of -40.21374498\,Km/s for an
energy incident beam of 0.72\,eV. This component and the corresponding layer by
layer contribution is depicted in Fig. \ref{fig:alt-vyz-lay}. From this figure
we have that for the energy range from 0.70\,eV to 0.74\,eV the fifth and sixth
layers corresponding to the bottom carbon and hydrogen numbered with 5 and 6 in
Fig. \ref{fig:alt-struc} have contributions in opposite direction than the other
4 layers resulting in a total response $\mathcal{V}^{\mathrm{yz}}= -40.2$\,Km/s
for an incoming beam energy of 0.72\,eV. In the other hand, for the energy range
from 0.88\,eV to 0.95\,eV the response for the all six layers the responses are
in the same direction resulting in a total response
$\mathcal{V}^{\mathrm{yz}}=-32.89$\,Km/s for an incoming beam with energy of
0.912\,eV.


%%%%%%%%%%%%%%%%%%%%%%%%%%%%%%%%%%%%%%%%%%%%%%%%%%%%%%%%%%%%%%%%%%%%%%%%%%%%%%
%%%%%%%%%%%%%%%%%%%%%%%%%%% Res: fixin vel Up  %%%%%%%%%%%%%%%%%%%%%%%%%%%%%%%


\subsection{Fixing spin} % (fold)
\label{sec:res-fixspin}



% subsection res-fixspin (end)


\bibliography{article.bib}

\end{document}


